\documentclass[a4paper]{article}

%% Language and font encodings
\usepackage[utf8]{inputenc}
\usepackage[T1]{fontenc}
\usepackage{float}
\usepackage{booktabs}
\usepackage{listings}
%\usepackage{listingsutf8}

%--------------------------------------
 
%Portuguese-specific commands
%--------------------------------------
\usepackage[portuguese]{babel}

%% Sets page size and margins
\usepackage[a4paper,top=3cm,bottom=2cm,left=3cm,right=3cm,marginparwidth=1.75cm]{geometry}

%% Useful packages
\usepackage{amsmath}
\usepackage{graphicx}
\usepackage[colorinlistoftodos]{todonotes}
\usepackage[colorlinks=true, allcolors=blue]{hyperref}
\usepackage{caption}
\usepackage{subcaption}
\usepackage{indentfirst}
\usepackage{pgfplots}   % pacote para uso do pgfplots
\usepackage{hyperref} % pacote para inserir links com \url{}
\usepackage{siunitx}
\usepackage{pgfplotstable}
\usepackage{xcolor}



\definecolor{mygrey}{gray}{0.95}
\definecolor{mygreen}{cmyk}{0 1 0.00005 0}
\definecolor{comment}{cmyk}{0.99998 1 0 0}
\definecolor{number}{gray}{0}
\definecolor{string}{cmyk}{0 0.99988 1 0}



\lstdefinestyle{mystyle}{
    backgroundcolor=\color{mygrey},   
    commentstyle=\color{comment},
    keywordstyle=\color{mygreen},
    numberstyle=\tiny\color{number},
    stringstyle=\color{string},
    basicstyle=\ttfamily\footnotesize,
    breakatwhitespace=false,         
    breaklines=true,                 
    captionpos=b,                    
    keepspaces=true,                 
    numbers=left,                    
    numbersep=5pt,                  
    showspaces=false,                
    showstringspaces=false,
    showtabs=false,                  
    tabsize=2,
    language=octave,
    literate=
        {á}{{\'a}}1
        {à}{{\`a}}1
        {ã}{{\~a}}1
        {â}{{\^a}}1
        {é}{{\'e}}1
        {ê}{{\^e}}1
        {í}{{\'i}}1
        {ó}{{\'o}}1
        {õ}{{\~o}}1
        {ú}{{\'u}}1
        {ü}{{\"u}}1
        {ç}{{\c{c}}}1
}

\lstset{style=mystyle}

\newcommand{\HRule}[1]{\rule{\linewidth}{#1}}



\title{MS211 - Cálculo Numérico - Turma J - Atividade 2
}
\author{Luiz Henrique Costa Freitas RA: 202403 \\
        Gabriel de Freitas Garcia RA: 216179 \\
        Vanessa Vitória de Arruda Pachalki RA: 244956
}

\begin{document}
\renewcommand{\abstractname}{Enunciado Atividade 2}
\maketitle

\begin{abstract}
Podemos utilizar os métodos para encontrar raízes de funções para estimar a raiz
quadrada de um número. Considere a função $f(x) = x^2 - A$, onde $A$ representa o número que queremos estimar a raiz quadrada. Vamos utilizar o método de Newton para tanto.
\begin{enumerate}
    \item Escreva um pseudocódigo para o método de Newton e implemente como uma função do Octave. \label{item1}
    \item Implemente a função $f(x)$ em Octave deixando $A$ como um parâmetro de entrada. \label{item2}
    \item Teste sua função para $A = 13$. Use $x_0 = 3$ como aproximação inicial e erro $\epsilon = 10^{-8}$\label{item3}
    \item Teste sua função para um valor de $A$ inteiro à sua escolha, desde que não tenha raiz inteira. Use o mesmo $\epsilon$ do item anterior. Como $x_0$ use um inteiro que
    deve ser a raiz quadrada de algum número menor que $A$. \label{item4}
    \item Repita esse experimento com o método das secantes. Utilize $x_1$ um inteiro que é a raiz quadrada de um número maior que $A$.\label{item5}
    \item Compare o comportamento dos dois métodos nos exemplos acima. \label{item6}
\end{enumerate}
   
\end{abstract}

\section{Respostas}

Para resolver o item~\ref{item1}, desenvolvemos o seguinte algoritmo: 
\begin{lstlisting}
f = minha_funcao(x)
f_linha = derivada(f)
x = x0
while(f(x) > erro):
	x = x - f(x)/f_linha(x)
\end{lstlisting}

A partir deste pseudocódigo e assumindo as funções $f(x)$ e $f^{'}(x)$, o valor inicial $x_0$ e o erro $\epsilon$ anteriormente definidas podemos representar este mesmo algoritmo em Octave:

\begin{lstlisting}[language = octave]
# função que calcula o loop do método de Newton
function res = newton_loop(x,A,f_linha,erro)
	k = 0;
	while(abs(f(x,A)) > erro)
		x = passo_newton(x,A,f_linha);
		k++;
		disp("Aproximação atual:");
		disp(x);
	endwhile
	res = x;
	disp("Número de passos:");
	disp(k);
endfunction

# função que calcula o passo de Newton

function x_res = passo_newton(x,A,f_linha)
	x_res = x - f(x,A)/f_linha(x);
endfunction
\end{lstlisting}

Para o item~\ref{item2} definimos a função $f(x)$ desta maneira:

\begin{lstlisting}[language = octave]
function res = f(x,A)
	res = x^2 - A;
endfunction
\end{lstlisting}

E com esta definição de função obtivemos o seguinte código para calcular o Método de Newton para a equação dada:
\begin{lstlisting}[language = octave]
# Jesus quer transformar sua vida
# Arquivo: lab02.m
format long;
A = input("Digite um número inteiro:\n");
f_linha = @(x) 2*x;
x = input("Digite o valor inicial:\n");
erro = input("Digite o valor de erro: \n"); 
x = newton_loop(x,A,f_linha,erro);
disp("x final:");
disp(x);
disp("f(x) final: ");
disp(f(x,A));
\end{lstlisting}

Com este programa obtivemos as repostas para o item \ref{item3}
\begin{lstlisting}
>> lab02

Digite um número inteiro:
13
Digite o valor inicial:
3
Digite o valor de erro:
10^-8
Aproximação atual:
3.666666666666667
Aproximação atual:
3.606060606060606
Aproximação atual:
3.605551311433664
Aproximação atual:
3.605551275463990
Número de passos:
4
x final:
3.605551275463990
f(x) final:
1.776356839400250e-15
\end{lstlisting}

Semelhantemente obtivemos o resultado para o item \ref{item4}:
\begin{lstlisting}
>> lab02
Digite um número inteiro:
47
Digite o valor inicial:
6
Digite o valor de erro:
10^-8
Aproximação atual:
6.916666666666667
Aproximação atual:
6.855923694779117
Aproximação atual:
6.855654605682009
Aproximação atual:
6.855654600401045
Número de passos:
4
x final:
6.855654600401045
f(x) final:
7.105427357601002e-15
\end{lstlisting}

Para a resolução do item \ref{item5} repetimos o experimento e obtivemos o seguinte algoritmo:
\begin{lstlisting}
x1 = algum_ponto
x2 = algum ponto_diferente
f = minha_funcao(x2)
g = (f(x2) - f(x1))/(x2-x1)
while(f(x2) > erro):
auxiliar = x2
x2 = x2 - f(x2)/g(x1,x2)
\end{lstlisting}

E o seguinte código em Octave:

\begin{lstlisting}[language = octave]
# código principal

# Jesus quer transformar sua vida
# arquivo: lab02p2.m
format long;
A = input("Digite um número inteiro:\n");
g = @(x1,x2) (f(x2,A) - f(x1,A))/(x2-x1);
x1 = input("Digite o valor inicial:\n");
x2 = input("Digite o segundo valor inicial: \n");
erro = input("Digite o valor de erro: \n"); 
x2 = sec_loop(x1,x2,A,g,erro);
disp("x final:");
disp(x2);
disp("f(x) final: ");
disp(f(x2,A));

#função que executa o loop do método das secantes

function res = sec_loop(x1,x2,A,g,erro)
	k = 0;
	while(abs(f(x2,A)) > erro)
		aux = x2;
		x2 = passo_sec(x1,x2,A,g);
		x1 = aux;
		k++;
		disp("Aproximação atual:");
		disp(x2);
	endwhile
	res = x2;
	disp("Número de passos:");
	disp(k);
endfunction

# função que calcula o passo do método das secantes

function x_res = passo_sec(x1,x2,A,g)
	x_res = x2 - f(x2,A)/g(x1,x2);
endfunction
\end{lstlisting}

E obtivemos os seguintes resultados:

\begin{lstlisting}
>> lab02p2

Digite um número inteiro:
13
Digite o valor inicial:
3
Digite o segundo valor inicial:
4
Digite o valor de erro:
10^-8
Aproximação atual:
3.571428571428572
Aproximação atual:
3.603773584905661
Aproximação atual:
3.605559729526671
Aproximação atual:
3.605551273379371
Aproximação atual:
3.605551275463987
Número de passos:
5
x final:
3.605551275463987
f(x) final:
-1.776356839400250e-14
>> lab02p2

Digite um número inteiro:
47
Digite o valor inicial:
6
Digite o segundo valor inicial:
7
Digite o valor de erro:
10^-8
Aproximação atual:
6.846153846153846
Aproximação atual:
6.855555555555555
Aproximação atual:
6.855654669078660
Aproximação atual:
6.855654600400548
Número de passos:
4
x final:
6.855654600400548
f(x) final:
-6.799893981224159e-12
\end{lstlisting}

Os dois métodos são muito parecidos entre si, porém o Método de Newton trabalha com a derivada da função enquanto o Método das Secantes utiliza a secante da função. Contudo, quanto menor a distância dos dois pontos $x$ no segundo método, mais ele se aproxima do valor da derivada, deixando os métodos bastante semelhantes. O método de Newton, portanto, pode ser entendido como um caso particular do Método das Secantes em que a distância dos dois pontos que cruzam a função é praticamente zero, ou seja, a derivada da função no ponto.

O primeiro método (Newton) costuma ser o mais eficiente e isso foi observado no item \ref{item3} e \ref{item4}. No item \ref{item3}, isso pode ser observado no número de iterações que cada método precisou para chegar no resultado final (Newton = 4 e Secantes =5). Já no item \ref{item4}, percebe-se que o número de iterações foi a mesma, entretanto, o erro apresentado no segundo método  foi maior (na ordem de $10^3$ a mais que o primeiro).

\end{document}
